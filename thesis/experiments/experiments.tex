\chapter{Experimental Setup} \label{chap:experiments}

In this chapter we detail the steps carried out as part of our investigation.
We first outline the nature of the data set, describe how it was cleaned and
pre-processed before it was used to train various classifiers. The methods of
evaluation are then presented, as these are critical to assessing the
performance of the models created.

\section{Trauma patient data set}

\subsection{Collection}
The data set we used consisted of trauma registry data from the trauma centre
at the Royal Prince Alfred Hospital, a major trauma centre in New South Wales,
Australia. It covers all adult (age 15 and over) inhospital admissions to the
trauma centre from 2007--2011. 

All patients are first admitted to the trauma ward until discharged
or transferred to an appropriate unit within the hospital. 

\subsection{Characteristics}

\section{Data pre-processing}

\subsection{Cleaning} % missing values, irrelevant features

\subsection{Normalisation}

\subsection{Feature selection}

\section{Model derivation}

\subsection{Single classifier}

\subsubsection{Parameter tuning}

\subsection{Ensemble classifiers}

\section{Performance evaluation}

\subsection{Metrics}

\subsection{Cross-validation}

\subsection{Statistical tests}
