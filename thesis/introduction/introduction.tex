\chapter{Introduction}

\section{Background}
Suppose that we have a (perhaps large) data set. It could consist of
demographic data taken from a census, details of transactions in a shopping
centre, or simply a collection of text documents. The term used to describe
the methods of analysis used to find unknown relationships and summarise
information in these data sets is \textit{data mining}. Broadly speaking,
data mining can be broken down into five areas (or \textit{tasks}) depending
on the aims of the person analysing the data:

\begin{enumerate}
\item Exploratory data analysis (EDA), which consists of simply exploring some
given data set without any particular goal or ideas of what to look for.
Methods in EDA usually involve interaction and visualisation, but this can
become difficult if there are too many data points;
\item Descriptive modelling, as its name suggests, seeks to describe the data
or the processes that generate it. The major areas here are clustering (finding
partitions of the data) and density estimation (finding the probability
distribution of the data);
\item Predictive modelling aims to create models that will allow the value of
some desired quantity or metric to be predicted based on some known or observed
values in the data set. The two major sub-areas of predictive modelling are
\textit{classification} and \textit{regression}, distinguished only by the
nature of the quantity we want to predict. If we want to predict whether
something belongs in a category, such as the disease of a patient, then we
are dealing with classification; otherwise, if we wish to predict a numeric
quantity, such as the value of the stock market at some future data, then
it is a regression problem;
\item Pattern and rule discovery, which covers anomaly detection (finding data
points which do not fit an expected trend, like detecting credit card fraud)
and association rule learning (discovering relationships between various items
in transaction databases, such as the food items purchased frequently together
in a supermarket);
\item Content retrieval, where the user wishes to find patterns in a data set
that are similar to the one he or she specifies. Search engines are a familiar
application of this area of data mining.
\end{enumerate}

\subsection{Problem Statement}

\subsection{Previous Work}

\section{Contributions}

\section{Results and Evaluation}
