\chapter{Introduction}

\section{Background}
Suppose that we have a (perhaps large) data set. It could consist of
demographic data taken from a census, details of transactions in a shopping
centre, or simply a collection of text documents. The term used to describe
the methods of analysis used to find unknown relationships and summarise
information in these data sets is \textit{data mining}. Broadly speaking,
data mining can be broken down into five areas (or \textit{tasks}) depending
on the aims of the person analysing the data:

\begin{enumerate}
\item Exploratory data analysis (EDA), which consists of simply exploring some
given data set without any particular goal or ideas of what to look for.
Methods in EDA usually involve interaction and visualisation, but this can
become difficult if there are too many data points;
\item Descriptive modelling, as its name suggests, seeks to describe the data
or the processes that generate it. The major areas here are clustering (finding
partitions of the data) and density estimation (finding the probability
distribution of the data);
\item Predictive modelling aims to create models that will allow the value of
some desired quantity or metric to be predicted based on some known or observed
values in the data set. The two major sub-areas of predictive modelling are
\textit{classification} and \textit{regression}, distinguished only by the
nature of the quantity we want to predict. If we want to predict whether
something belongs in a category, such as the disease of a patient, then we
are dealing with classification; otherwise, if we wish to predict a numeric
quantity, such as the value of the stock market at some future data, then
it is a regression problem;
\item Pattern and rule discovery, which covers anomaly detection (finding data
points which do not fit an expected trend, like detecting credit card fraud)
and association rule learning (discovering relationships between various items
in transaction databases, such as the food items purchased frequently together
in a supermarket);
\item Content retrieval, where the user wishes to find patterns in a data set
that are similar to the one he or she specifies. Search engines are a familiar
application of this area of data mining.
\end{enumerate}

Data mining overlaps substantially with machine learning, a sub-field of
artificial intelligence (AI) dedicated to the study of systems that can learn
from data rather than executing explicitly programmed instructions. The
techniques and issues discussed in this thesis are equally relevant in both
areas, so we will not attempt to define a clear distinction here.

In this thesis we will focus on the task of predictive modelling. Specifically,
our problem will be a classification one. We will introduce the necessary
terminology in the next section as we describe the problem.

\subsection{Problem Statement}
Consider a data set of patients admitted to the trauma ward of a hospital.
This data set looks like a very large table: there are many columns, each one
describing some aspect of a patient (for example, age); there are also many
rows, each corresponding to a particular patient that was admitted. We will
refer to the columns as \textit{features} or \textit{attributes}, so-named
because they describe some particular aspect of the rows, which we will call
\textit{samples}. Concretely, as an example, our data set could contain
two features (age and gender) for five patients: as a table, this means that
we have two columns, one for age and another for gender, with five rows
containing the age and gender data for each patient. Attributes can be numeric
(such as age) and potentially take on an infinite number of different values,
or categorical (such as gender) and can only have one of a limited, pre-defined
list of values.

In a general predictive modelling problem, we wish to predict a value for some
desired quantity given some values for the attributes. In our problem, we wish
to predict the length of stay (LOS) for a patient admitted to a hospital trauma
ward. To do this, we use our data set of trauma patients with \textit{known}
values for the LOS and apply various predictive modelling techniques to deduce
an effective model that will be able to predict the LOS of patients in advance.
Note that the LOS is simply another attribute of our data set; however, because
it is the quantity that we are interested in predicting, it is also called the
\textit{target} or \textit{outcome} variable.

Like the other attributes, the target or outcome variable can also be numeric
or categorical. This determines the sub-task of predictive modelling that we
will subsequently be dealing with: regression if the target variable is
numeric, and classification if it is categorical. The various values that the
categorical target variable can take on are also called \textit{classes}.
Our problem is one of classification, divided into two sub-problems:

\begin{enumerate}
\item The first problem is a \textit{binary} classification task in which we
predict one of two possible outcomes (or classes) for the patient: LOS less
than 2 days, or greater than 2 days. Given this, what model performs best?
\item The second problem is a \textit{multi-class} classification in which we
define several possible outcomes with different ranges for the LOS and predict
which range the patient will fall into. An example of possible outcomes can be
1-3 days, 4-10 days, and 11 or more days, giving us three classes for the
target variable. The optimal division of LOS into ranges will be investigated
as part of the problem.

\subsection{Previous Work}

\section{Contributions}

\section{Results and Evaluation}
