\documentclass[portrait]{usydposter}
\usepackage{xspace}

\newcommand{\acronym}[1]{\textsc{#1}\xspace}
\newcommand{\cf}[1]{\mbox{$\it{#1}$}}
\newcommand{\todo}[1]{{\color{red} #1}}

\newcommand{\ngram}{n-gram\xspace}
\newcommand{\ngrams}{{\ngram}s\xspace}
\newcommand{\candc}{C\&C\xspace}
\newcommand{\ccgbank}{CCGBank\xspace}
\newcommand{\ccg}{\acronym{ccg}}
\newcommand{\cky}{\acronym{cky}}
\newcommand{\nlp}{\acronym{nlp}}
\newcommand{\np}{\acronym{np}}
\newcommand{\pos}{\acronym{pos}}
\newcommand{\wsj}{\acronym{wsj}}

\flushbottom

\title{Improved Prediction of Hospital Length of Stay for Severe Injury}
\author{Tianyu Pu \xspace \texttt{tianyu.pu@sydney.edu.au}}

\begin{document}

\makeheader

\begin{multicols}{3}

% =============================================================================
\section{Problem}
\noindent There are limited beds in hospital trauma wards, and yet there is a
constant demand for these beds by the inflow of severely injured patients. Many
patients occupy these beds when they could be better suited in another
specialised ward. If we could accurately predict how long an injured patient
would need to be hospitalised, we could make a more informed decision to place
them in another ward, freeing up these beds for the urgently injured who need
them most.

% =============================================================================
\section{Motivation}
\noindent Accurate prediction of the hospital length of stay in various medical
domains has been extensively studied, as it is a key indicator of the level of
resource utilisation in hospitals and directly impacts the use of their limited
funding \cite{Walczak2003}.

% =============================================================================
\section{Contribution}
\noindent

% =============================================================================
\section{Dataset}

% =============================================================================
\section{Approach}

% =============================================================================
\section{Evaluation Metrics}

% =============================================================================
\section{Results}

% =============================================================================
\section{Conclusions and Future Work}

% =============================================================================

\references
\bibliography{../thesis/references}

\end{multicols}
\end{document}
