\documentclass[a4paper]{article}
\usepackage{hyperref}
\usepackage[utf8]{inputenc}
\usepackage{natbib}

\title{INFO5993 IT Research Methods \\
    Assignment 2: Outline of Research Approach}
\author{Tianyu Pu \\
    Supervisor: Irena Koprinska}

\date{\today}

\begin{document}

\maketitle

\section{Research Contribution}
\subsection{Background}
Hospitals are always subject to limitations in funding, and as a result,
it is critical that they effectiely utilise the resources at their
disposal.
A key measure of hospital activity and utilisation is the length of stay
of patients, usually expressed in days. To this end, much research has been
conducted in the past few decades to find accurate and reliable models to
assist in predicting the length of stay of patients, so as to allow hospitals
to allocate their resources and plan their treatment processes more
efficiently. This is especially important in settings where there is a
prominent scarcity of a resource, such as patient beds\citep{Tu1993}.

Due to the numerous functions and units of hospitals, and the absence of a
standard in medical data collection, many models have been devised in the
literature applying statistical or data mining techniques (or both) to
length of stay prediction for various medical domains, such as burn injuries
\citep{Yang2010} and psychiatry\citep{Lowell1997}. Among these, logistic
regression remains a commonly-used and well-understood technique
\citep{Gabbe2005}\citep{Tu1996}. Artificial neural networks are also gaining
attention in the medical domain through their use in predicting various
outcomes of diseases such as breast cancer\citep{Bellazzi2008}. However,
many works have restricted themselves to looking at a few statistical or data
mining methods in a very specific domain.

\subsection{Intended Contribution}
As part of my research, I intend to:
\begin{enumerate}
\item apply a range of data mining techniques (artificial neural
networks, support vector machines, decision trees, na\"{i}ve Bayesian
classifier, Bayesian networks) to predict the length of stay of patients that
are admitted to a trauma centre. This builds upon the work of Dinh et al.
\citep{Dinh2013a}, who constructed a logistic regression model;
\item develop a new, accurate, and most importantly reliable method for
predicting the length of stay of trauma patients, based on the advantages and
drawbracks of the investigated data mining methods
\item use data mining techniques to discover what patient factors contribute
to their transfer to rehabilitation after trauma, and to predict the
probability of requiring rehabilitation upon admission;
\item generalise my findings in the trauma domain into other hospital functions
to see if there are over-arching, universal, or common factors that contribute
to how long patients stay in hospital; and
\item develop a mobile application that is easy-to-use and reliable that will
incorporate the proposed methods to assist medical professionals in resource
planning.
\end{enumerate}

The value of the work will be multi-faceted: accurately predicting how long
a patient will stay when they are first admitted will allow hospital staff to
better plan their resources (such as beds, medication, and the staff
themselves) to efficiently use the funding that they receive; additionally,
knowing about common warning signs across all medical domains that indicate
longer stays means that staff can plan for long-term structural improvements
in the hospital (for example, if a particular ward receives patients that
characteristically require longer-term care, then they could look at adding
more beds or other decisions to improve the delivery of care to that ward).

\section{Evaluation}
In order to gauge how the data mining models perform, I will compare them to
the logistic regression model that Dinh et al.\citep{Dinh2013a} came up with.
In that study, the area under the receiver operating characteristic curve was
used to measure the performance of the model, and I will also use this
measurement to evaluate the models that I construct. Additionally, I will
also use a number of other metrics (such as sensitivity, specificity,
classification accuracy, and mean absolute difference\citep{Walczak2003}) to
evaluate and compare the performance of the various techniques.

It is also important to keep the training and testing data separate as part
of the evaluation process. To this end, I will use ten-fold cross-validation
and the average of the runs will be reported. In order for this to be
comparable to Dinh et al.\citep{Dinh2013a}, who used bootstrap
cross-validation, I will re-run their method and use ten-fold cross-validation
to evaluate the performance of their model. These techniques ensure that we
obtain an accurate record of the performance of the constructed models, and
allows us to make comparisons and draw conclusions and insights from them.

\bibliographystyle{abbrv}
\bibliography{../hons/references}

\end{document}
